% Geometry, font
\documentclass[12pt, letter]{article}
\usepackage[margin=0.8in]{geometry}
\usepackage[T1]{fontenc}
\usepackage{fourier}
\usepackage{titling}
\setlength{\droptitle}{-5em} 
\usepackage[parfill]{parskip}
\usepackage{graphicx}
\graphicspath{{imgs/}}
\usepackage{hyperref}

% Math stuff
\usepackage{amssymb}
\usepackage{bm}

% Code Highlighting
\usepackage{minted}
\usemintedstyle{solarizedlight}

\author{Zach Neveu}
\title{ Day 10 Notes }

\begin{document}
\maketitle

\section{Agenda}%
\label{sec:agenda}
\begin{itemize}
	\item Quiz
	\item LP Standard Form
	\item More LP examples
	\item Solving LP Problems
\end{itemize}

\section{LP Examples}%
\label{sec:lp_examples}

\subsection*{Personnel Scheduling Problem}
\begin{itemize}
	\item Goal: find an assignment of people to shifts such that enough people are working in each slot and total cost is minimized.
	\item Objective function: $z = 170x_1+160x_2+175x_3+180x_4+195x_5$
	\item Constraint example: $x_1+x_2 \ge 79$
	\item Constraint form:  $\sum$ active shifts during period $\ge$ required personnel during period.
	\item Following constraints for all time periods create many redundant constraints
	\item \textbf{redundant constraint: } Constraint can be deleted without changing the solution to the LP
	\item Solution to this problem: x=(48,31,39,43,15), z=30,610
\end{itemize}
\begin{table}[h]
	\centering
	\caption{Personnel Schedules}
	\label{tab:schedule}
	\begin{tabular}{|l|c|c|}
	\hline
	Shift & Time Range & Hourly Wage \$\\
	\hline
	1 & 6a-2p  & 170 \\
	\hline
	2 & 8a-4p  & 160 \\
	\hline
	3 & 12p-8p & 175 \\
	\hline
	4 & 4p-12a & 180 \\
	\hline
	5 & 10p-6a & 195 \\
	\hline
	\end{tabular}
\end{table}

\begin{table}[h]
	\centering
	\caption{People Required}
	\label{tab:numpeople}
	\begin{tabular}{|c|c|}
	\hline
	Shift & People Needed \\
	\hline
	6-8   & 48 \\
	8-10  & 79 \\
	10-12 & 65 \\
	12-2  & 87 \\
	2-4   & 64 \\
	4-6   & 73 \\
	6-8	  & 82 \\
	8-10  & 43 \\
	10-12 & 52 \\
	12-6a & 15 \\
	\hline
	\end{tabular}
\end{table}

\section{LP Standard Form}%
\label{sec:lp_standard_form}
\begin{itemize}
	\item Maximize $\sum_{j=1}^{n} c_jx_j$ subject to $ \sum_{j=1}^{n} a_{ij}x_j \le b_i, i=1:m$, $x_j \ge 0$
	\item Always maximizing
	\item Constraints always use $\le$
	\item All variables $\ge 0$
	\item Any LP problem can be rephrased into a standard form equivalent
	\item \textbf{Equivalent:} Two LP formulations are equivalent if they are both maximizing and for every feasible solution in L there is a corresponding feasible solution in L' with the same objective  value and visa versa.
	\item If $L$ is minimizing and $L'$ is maximizing, multiply objective function by $-1$ 
	\item If not all variables are constrained by $\ge 0$ : create 2 positive variables and take their difference to get the unbounded value (e.g. $x_2 \rightarrow (x_2'-x_2'')$)
	\item If equality used instead of inequality in constraint: Split into $\le$ and $\ge$ constraints, and flip $\ge$ one.
\end{itemize}
\end{document}
