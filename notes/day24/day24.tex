% Geometry, font
\documentclass[12pt, letter]{article}
\usepackage[margin=0.8in]{geometry}
\usepackage[T1]{fontenc}
\usepackage{fourier}
\usepackage{titling}
\setlength{\droptitle}{-5em} 
\usepackage[parfill]{parskip}
\usepackage{graphicx}
\graphicspath{{imgs/}}
\usepackage{hyperref}

% Math stuff
\usepackage{amssymb}
\usepackage{amsmath}
\usepackage{bm}

% Code Highlighting
\usepackage{minted}
\usemintedstyle{solarizedlight}

\author{Zach Neveu}
\title{ Day 24 Notes }

\begin{document}
\maketitle
\section{Agenda}%
\label{sec:agenda}
\begin{itemize}
	\item Quiz
	\item Max-Flow Review
	\item Max-Flow Min-cut theorem
\end{itemize}

\section{Max-Flow Review}%
\label{sec:max_flow_review}
\begin{itemize}
	\item Ford-Fulkerson - while an augmenting path exists, add flow to it
	\item How do we know this is optimal?
	\item \textbf{cut}: a slice across a flow-network dividing the network into two parts with the source and sink on opposite sides (s,t). Capacity of a cut is $c(s,t)$ which is the net flow across the cut from the source to the sink.
	\item Flow over a cut can't be larger than capacity of cut.
	\item Overall flow limited by minimum cut.
	\item Hypothetically, if we found a max flow, there will be no augmenting path.
	\item If the flow has hit the capacity for some cut, then the capacity has been hit.
	\item If a max-flow has been found, then a cut through the residual capacity diagram exists where remaining capacity is 0.
\end{itemize}

\section{Ford-Fulkerson Efficiency}%
\begin{itemize}
	\item Number of iterations needed upper-bounded by value of max flow (flow increases at least 1 on every iter)
	\item Overall efficiency: $O(Ef*)$. Depends on size of maximum flow
	\item Other algorithms
	\begin{itemize}
	 	\item Edmonds-Karp: $O(VE^2)$
		\item Also algorithms for $O(V^2E)$, $O(V^{3})$
	\end{itemize}
\end{itemize}


\end{document}
