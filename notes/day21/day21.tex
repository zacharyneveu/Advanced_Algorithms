% Geometry, font
\documentclass[12pt, letter]{article}
\usepackage[margin=0.8in]{geometry}
\usepackage[T1]{fontenc}
\usepackage{fourier}
\usepackage{titling}
\setlength{\droptitle}{-5em} 
\usepackage[parfill]{parskip}
\usepackage{graphicx}
\graphicspath{{imgs/}}
\usepackage{hyperref}

% Math stuff
\usepackage{amssymb}
\usepackage{amsmath}
\usepackage{bm}

% Code Highlighting
\usepackage{minted}
\usemintedstyle{solarizedlight}

\author{Zach Neveu}
\title{ Day 21 Notes }

\begin{document}
\maketitle
\section{Agenda}%
\label{sec:agenda}
\begin{itemize}
	\item Quiz
	\item Simulated Annealing
	\item Genetic Algorithms
	\item Reading
	\item Next Quiz Monday
	\item Homework Thursday
\end{itemize}

\section{Simulated Annealing}%
\label{sec:simulated_annealing}
\begin{itemize}
	\item Review of details: probability of accepting solution. 1 if better, less than 1 if worse.
	\item Results from last class: very slow, not awesome results
	\item How to speed up?
	\item Long time needed at each temperature
	\item Smaller neighborhood lowers time needed at each temperature
	\item Neighborhood pruning: lowers size of 2-opt neighborhood $O(n^2)\to O(n)$
	\item Low-temp start with good initial solution - reduces number of steps needed in annealing 50\% of neighbors accepted initially instead of 95\%.
	\item Low-temp start + neighborhood pruning $\to$ SA2
	\item SA2 results in table \ref{tab:sa}
	\item Still quite slow, but can beat LK given enough time
	\item No bound needed, no gradient needed, very flexible algorithm
	\item Further Variations of SA
	\begin{itemize}
		\item Adaptive scheduling technique: have temp lengths adapt to solution. For example, when champion solution isn't changing often, reduce the temperature.
		\item Water level algorithms - instead of looking at relative goodness of neighboring solution compared to current solution, compare neighbor to a global threshold or "water level". Raise the water level over time to anneal. Idea: imagine tides here! The search is like a sandpiper, explores when tide goes out, then runs back to best solution when tide comes in.
	\end{itemize}
\end{itemize}

\begin{table}[h]
	\centering
	\caption{Variants of SA and performance}
	\label{tab:sa}
    \begin{tabular}{|c|c|c|c|}
    \hline
    N= & 100 & 316 & 1000 \\
	\hline
    SA1 & 5.2(12.4s) & 4.1(188s) & 4.1(3170s) \\
    \hline
    SA1+2-opt & 3.4 &  3.7 & 4.0 \\
    \hline
	SA2 $\alpha=10$ & 1.6 (14.3s) & 1.8 (50.35s)& 2.0 (2290s) \\
	\hline
	SA2 $\alpha=40$ & 1.3 (58s) & 1.5 (204)& 1.7 (80509s) \\
	\hline
	SA2 $\alpha=100$ & 1.1 (141s) & 1.3 (655s)& 1.6 (191000s) \\
	\hline
    2-opt & 4.5(0.03s) & 4.8  (0.09s) & 4.9 (0.34s) \\
    \hline
    3-opt & 2.5(0.04s) & 2.5 (0.1s) & 3.1 (0.4s) \\
    \hline
    LK & 1.5(0.06) & 1.7(0.2) & 2.0(0.77s) \\
    \hline
    \end{tabular}
\end{table}

\section{Genetic Algorithms}%
\begin{itemize}
	\item Famous example of how nature has inspired algorithms
	\item Includes local search, intensification/diversification, randomization, but does so in a unique way.
	\item Idea: population of organisms. Pairs of organisms reproduce, individual organisms mutate, offspring share characteristics with parents, and rates of survival/reproduction tied to "fitness".
	\item Translation to optimization
	\begin{itemize}
		\item Collection of solutions. This is different from other solvers!
		\item Pairs of solutions produce new solutions
		\item Individual solutions change
		\item New solutions are similar to their parents
		\item Survival and reproduction related to objective value
	\end{itemize}
\end{itemize}

\begin{minted}{Python}
# Genetic Agorithm
pop = generate_pop()
done = False
while not done:
    subs = select(subset(pop, size=[1,2]))
    children = [mutate(x) for x in subs if x.size ==  1]
    children += [crossover(x[0],x[1]) for x in subs if x.size ==  2]
    survivors = find_survivors(children)
return maximum([fitness(x) for x in survivors])
\end{minted}

\begin{itemize}
    \item Required features
    \begin{itemize}
        \item Population size
        \item How to select initial population
        \item Mating strategy
        \item Mutation strategy
        \item Crossover strategy
        \item Survival function
    \end{itemize}
    \item Initial population
    \begin{itemize}
        \item Heuristics for best solutions
        \item Random
        \item Should have some diversity
    \end{itemize}
    \item Mating Strategy
    \begin{itemize}
        \item Select randomly, in proportion to fitness
    \end{itemize}
    \item Mutation
    \begin{itemize}
        \item Solutions represented as string of digits, often bits.
        \item Mutation, is pick a random bit and flip it
        \item Can end up with unviable children not in F
    \end{itemize}
    \item Crossover
    \begin{itemize}
        \item Pick a crossover index in strings, choose first half of one, second half of the other
        \item idea: why not flip a coin on each bit?
        \item idea: what about probabilistic kind of truth table for each bit? i.e. if parents are both 1, child is 1 with p=0.99.
    \end{itemize}
    \item Selection
    \begin{itemize}
        \item Keep with probability proportional to fitness
        \item Only keep offspring
    \end{itemize}

\end{itemize}
\end{document}
