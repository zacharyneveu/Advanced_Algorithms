% Geometry, font
\documentclass[12pt, letter]{article}
\usepackage[margin=0.8in]{geometry}
\usepackage[T1]{fontenc}
\usepackage{fourier}
\usepackage{titling}
\setlength{\droptitle}{-5em} 
\usepackage[parfill]{parskip}
\usepackage{graphicx}
\graphicspath{{imgs/}}
\usepackage{hyperref}

% Math stuff
\usepackage{amssymb}
\usepackage{bm}

% Code Highlighting
\usepackage{minted}
\usemintedstyle{solarizedlight}

\author{Zach Neveu}
\title{ Day 16 Notes }

\begin{document}
\maketitle

\section{Agenda}%
\label{sec:agenda}
\begin{itemize}
	\item Knapsack branch and bound
	\item Project \#4 Introduction
	\item Intro to Dynamic Programming
	\item Quiz Tomorrow
	\item HW due Thursday 6/6
	\item Project 4 due Monday 6/10
	\item Quiz Feedback
\end{itemize}

\section{Knapsack Branch and Bound}%
\label{sec:knapsack_branch_and_bound}
\begin{itemize}
	\item Recall example from last class
	\item Key Ideas
	\begin{itemize}
		\item Branch on each item in knapsack
		\item Estimate bound by filling remaining space with ratio of next highest item
		\item This bound is somewhat loose - obviously not all items can be this good
	\end{itemize}
	\item Improved bound: Add available items, when item won't fit, include the largest fraction of it that will!
	\item This is just like the LP bound on ILP!
	\item This knapsack problem, where items can be partially included is called \textbf{Fractional Knapsack} 
	\item Solving fractional knapsack gives tight bound on solving 0-1 knapsack.
\end{itemize}

\section{Project 4 Introduction}%
\label{sec:project_4_introduction}
\begin{itemize}
	\item 2 Part project
	\item Given header file for knapsack object
	\item implement \texttt{Knapsack::bound()} which finds the improved bound above
	\begin{itemize}
		\item Beware special cases!
	\end{itemize}
	\item Implement \texttt{Knapsack::branchandbound()} searches for optimal solution using \texttt{bound()} 
	\begin{itemize}
		\item Decide how to branch on variables
		\item Decide order to divide subproblems
		\item Use knapsack object to store subproblems - each subproblem gets a knapsack
		\item Use variable \texttt{num} to indicate how many variables have been fixed
		\item Suggestion: use list structure (stl::deque) of subproblems. Choose item from deque, expand it, add new subproblems back to deque.
		\item When feasible solns found keep track of the best one
		\item 10 Minute limit
	\end{itemize}
\end{itemize}

\section{Dynamic Programming}%
\label{sec:dynamic_programming}
\begin{itemize}
	\item \textbf{Mysterious}
	\item NOT related to dynamic memory stuff
	\item Advanced algorithm design: make sure you are doing everything exactly once, not less, not more.
	\item \textbf{Dynamic Programming} is about not duplicating work
	\item Requires very specific properties. Rarely useful, but very good when it works.
	\item Revolves around Multiplying N matrices \{a1, a2, a3, \ldots, aN\}
	\item What order do multiplications go? For N=3, can use (A1*A2)A3 or A1(A2*A3).
	\item Order doesn't effect result, but does effect work required!
	\item Assume multiplying $pxq$ and $qxr$ matrices to get $pxr$.
	\item Final result is computing $p*r$ values
	\item Multiplications generally much slower than additions
	\item $q$ products required to compute each entry of result
	\item $pqr$ total products required - $n^3$ time, kinda slow
\end{itemize}

\begin{gather*}
A_1 \rightarrow 10 x 100 \\
A_2 \rightarrow 100 x 5 \\
A_3 \rightarrow 5 x 50 \\
(A_1A_2)A_3 \rightarrow (10*100*5)+(10*5*50) = 7500 \\
A_1(A_2A_3) \rightarrow (100*5*50)+(10*100*50) = 75,000 \\
\end{gather*}

\begin{itemize}
	\item One answer is 100x harder to get!!
	\item How to minimize the work done?
	\item Number of ways to compute - huge for large N
	\item \textbf{Parenthesization} - number of ways to add parentheses to group values
\end{itemize}

\begin{gather*}
A_1,A_2,A_3,A_4, \ldots, A_{n-1}, A_n \\
p_0 x p_1, p_1 x p_2, p_2 x p_3, \ldots, p_{n-1} x p_n \\
A_{i..j} = A_iA_{i+1}A_j \rightarrow \text{Partial Product}
\end{gather*}

\begin{itemize}
	\item Assume an optimal parenthesization of some product has been found
	\begin{enumerate}
		\item There must be a final product to compute
		\item Total cost of that optimal parenthesization = (cost to compute final product + price to compute LHS + price to compute RHS)
		\item \textbf{Optimal Substructure Property}: Optimal parenthesization of entire problem must contain optimal parenthesizations of subproblems. If LHS+RHS+FP is optimal, then LHS \& RHS must both be optimal as well.
	\end{enumerate}
	\item Notation
	\begin{itemize}
		\item Let $m[i,j] = $ Minimum \# of multiplications needed to compute subproduct $A_{i..j}$ 
		\item Looking for $m[1,n]$ eventually
		\item Final multiplication:  $A_{i..k}*A_{k+1..j}$.
		\item $m[i,j] = m[i,k] + m[k+1,j] + p_{i-1}p_kp_j$
		\item Problem: optimal $k$ unknown\ldots
		\item Try all values of k and pick the best! Only $O(n)$
		\item $m[i,j] = 0 if (i == j)$, else $argmin_k(m[i,k]+m[k+1,j]+p_{i-1}p_kp_j) \\$
	\end{itemize}
\end{itemize}

\end{document}
